\documentclass[11pt,a4paper]{article}

\usepackage{amsmath}
\usepackage{amsthm}
\usepackage{amssymb}
\usepackage{booktabs}
\usepackage{graphicx}
\usepackage{color}
\usepackage{fullpage}
\usepackage{listings}
\usepackage{color}
\usepackage{url}
\usepackage{multirow}
\usepackage{float}
\usepackage{hyperref}

% for Chinese
\usepackage{fontspec}
\usepackage[BoldFont, SlantFont]{xeCJK}
% [字型設定]
\setCJKmainfont{Noto Serif CJK TC} 
% 其他備選字型 (若編譯失敗可嘗試切換):
% \setCJKmainfont{PingFang TC}       % MacOS
% \setCJKmainfont{Heiti TC}          % MacOS
% \setCJKmainfont{Microsoft JhengHei}% Windows
% \setCJKmainfont{FandolSong}        % TeX Live 內建 (Linux/Overleaf)

\renewcommand{\baselinestretch}{1.3}

\parskip=5pt
\parindent=22pt
\newtheorem{lemma}{Lemma}
\newtheorem{ques}{Question}
\newtheorem{prop}{Proposition}
\newtheorem{defn}{Definition}
\newtheorem{rmk}{Remark}
\newtheorem{note}{Note}
\newtheorem{eg}{Example}
\newtheorem{aspt}{Assumption}

\definecolor{emphOrange}{RGB}{247, 80, 0}
\definecolor{stringGray}{RGB}{109, 109, 109}
\definecolor{commentGreen}{RGB}{0, 96, 0}
\definecolor{mygreen}{rgb}{0,0.6,0}
\definecolor{mygray}{rgb}{0.5,0.5,0.5}
\definecolor{mymauve}{rgb}{0.58,0,0.82}

\lstset{
  belowcaptionskip=1\baselineskip,
  breaklines=true,
  frame=L,
  language = SQL,
  showstringspaces=false,
  basicstyle = \ttfamily\small, 
  keywordstyle = \bfseries\color{blue}, 
  emph = {symbol1, symbol2},
  emphstyle = \color{red},
  emph = {[2]symbol3, symbol4},
  emphstyle = {[2]\color{emphOrange}},
  commentstyle = \color{commentGreen}, 
  stringstyle = \color{stringGray}, 
}

\renewcommand{\abstractname}{\bf 摘要}
\renewcommand{\tablename}{表}
\renewcommand{\figurename}{圖}
\renewcommand{\refname}{\bf 參考文獻}

\begin{document}

\title{}
\author{}
\date{}

\begin{center}
\textbf{\Large 資料庫管理(114-1) \\ [5pt]
期末專案完整報告} \\ [10pt] 
第 X 組 \ % 請填入組別
學號 姓名、學號 姓名 \ % 請填入組員資訊
2025 年 12 月 3 日 \ 
\href{https://github.com/pollop123/zoo_db}{GitHub 專案連結}、 \href{YOUR_VIDEO_LINK}{展示影片連結}
\footnote{展示影片的需求請以 NTU COOL 作業區的「期末專案任務敘述」為準。}
 \ 
\end{center}

\section{系統分析}

\subsection{系統介紹}

「ZooKeeper Pro」是一個專為現代化動物園設計的綜合管理系統,旨在解決傳統紙本記錄在資料保存、檢索與即時性上的不足。動物園的日常營運涉及大量且異質的數據,包括動物的健康狀況(如體重變化)、飲食紀錄、飼料庫存管理,以及員工的排班與證照管理等。

本系統採用 SQL 與 NoSQL 混合架構,利用 PostgreSQL 處理結構化且關聯性強的核心業務資料(如員工、動物、餵食紀錄),並引入 MongoDB 處理高寫入頻率且結構多變的日誌型資料(如稽核日誌與健康警示)。透過 CLI (Command-Line Interface) 介面,照養員可以快速回報動物狀況,管理員則能即時監控園區異常,實現數據驅動的智慧化管理。

根據權限不同,系統使用者分為兩種身分:User(照養員)及 Admin(管理員)。照養員主要負責現場的數據蒐集與回報,而管理員則負責資源調度、異常處理與系統維護。

\subsection{系統功能} 
\label{subsection:system functions}

\subsubsection{給 User (照養員) 的功能}

在本系統中,User 可以執行以下功能:
\begin{enumerate}
\item \textbf{新增餵食紀錄}:照養員在餵食動物後,需輸入動物 ID、飼料種類及數量。系統會即時檢查庫存,若庫存充足則扣除數量並記錄餵食時間。
\item \textbf{新增身體資訊}:定期量測並回報動物體重及健康狀態(如:正常、生病、受傷)。系統會自動比對歷史數據,若發現異常波動(如體重驟降)會即時發出警示。
\item \textbf{查詢班表}:查詢自己近期的排班資訊,包括負責的動物與工作項目。
\item \textbf{修正個人紀錄}:若照養員發現剛才輸入的資料有誤(例如輸入錯誤的體重),可在一定時間內進行修正。系統會保留修正前的數值作為稽核紀錄。
\item \textbf{查詢個別動物趨勢}:可查看特定動物近期的體重變化曲線與餵食歷程,輔助判斷動物健康狀況。
\end{enumerate}
   
\subsubsection{給 Admin (管理員) 的功能}

在本系統中,Admin 擁有所有 User 的功能,並額外包含:
\begin{enumerate}
\item \textbf{庫存管理}:查看所有飼料的庫存水位,並進行進貨操作。
\item \textbf{指派工作/排班}:新增或修改員工的排班紀錄。系統會自動檢查該員工是否具備照顧該特定動物所需的專業證照(如:猛獸專家、企鵝專家)。
\item \textbf{批量異常掃描}:一鍵掃描全園區所有動物的近期體重變化,列出變化率超過閾值(如 5%)的高風險清單。
\item \textbf{稽核日誌查看}:查詢所有資料修正紀錄,找出頻繁出錯的員工(冒失鬼名單)或異常的資料更動。
\item \textbf{員工與證照管理}:新增員工帳號,或授予員工特定的專業證照。
\end{enumerate}

\section{系統設計}

\subsection{ER Diagram}

圖 \ref{fig:erDiagram} 是「ZooKeeper Pro」的 ER Diagram。系統包含五個主要實體:\verb|EMPLOYEE|(員工)、\verb|ANIMAL|(動物)、\verb|FEEDS|(飼料)、\verb|TASK|(工作項目)以及 \verb|STATUS_TYPE|(狀態類別)。

\begin{itemize}
    \item \verb|EMPLOYEE| 與 \verb|ANIMAL| 之間存在多對多的 \verb|FEED|(餵食)關係,記錄誰餵了哪隻動物吃什麼。
    \item \verb|EMPLOYEE| 與 \verb|TASK| 之間存在多對多的 \verb|SHIFT|(排班)關係,每一次排班可指定負責的特定 \verb|ANIMAL|。
    \item \verb|ANIMAL| 擁有多筆 \verb|STATE_RECORD|(狀態紀錄),由 \verb|EMPLOYEE| 記錄。
    \item \verb|FEEDS| 擁有多筆 \verb|INVENTORY_LOG|(庫存變動),記錄進貨或餵食消耗。
\end{itemize}

% 請自行插入圖片,這裡使用 placeholder
%\begin{figure}[hbt]
%\centering
%\includegraphics[width=\textwidth]{figures/ER_diagram} 
%\caption{ZooKeeper Pro ER Diagram}
%\label{fig:erDiagram}
%\end{figure}

\subsection{Relational Database Schema Diagram}
\label{subsection:schema}

本系統採用 PostgreSQL 實作關聯式資料庫,主要資料表如下:

\begin{itemize}
    \item \verb|employee| (e\_id, e\_name, role, status, password\_hash)
    \item \verb|animal| (a\_id, species, sex, life\_status, required\_skill)
    \item \verb|feeds| (f\_id, feed\_name, category)
    \item \verb|feeding_records| (feeding\_id, a\_id, f\_id, fed\_by, amount, date)
    \item \verb|feeding_inventory| (stock\_entry\_id, f\_id, quantity\_delta, reason, date)
    \item \verb|animal_state_record| (record\_id, a\_id, weight, state\_id, recorded\_by, date)
    \item \verb|employee_shift| (shift\_id, e\_id, t\_id, a\_id, start\_time, end\_time)
    \item \verb|employee_skills| (e\_id, skill\_name)
\end{itemize}

此外,我們使用 MongoDB 儲存非結構化資料:
\begin{itemize}
    \item \verb|audit_logs|: 記錄資料修正與敏感操作的詳細軌跡。
    \item \verb|health_alerts|: 記錄系統自動偵測到的動物健康異常警示。
    \item \verb|login_logs|: 記錄員工的登入嘗試與結果。
\end{itemize}

\subsection{Data Dictionary}

以下列出部分關鍵資料表的欄位定義。

% ANIMAL
\begin{table}[H]
    \centering
    \resizebox{1\columnwidth}{!}{
    \begin{tabular}{llllll}
    \toprule
        Column Name & Meaning & Data Type & Key & Constraint & Domain \\ 
    \midrule
        a\_id & 動物編號 & varchar(10) & PK & Not Null & \\ 
        species & 物種 & varchar(50) & ~ & Not Null & ~ \\ 
        required\_skill & 所需技能 & varchar(50) & ~ & ~ & General, Carnivore... \\ 
        life\_status & 存活狀態 & varchar(20) & ~ & Not Null & In\_Zoo, Deceased \\ 
    \bottomrule
    \end{tabular}}
    \caption{資料表 ANIMAL 的欄位資訊} 
\end{table}

% FEEDING_RECORDS
\begin{table}[H]
    \centering
    \resizebox{1\columnwidth}{!}{
    \begin{tabular}{llllll}
    \toprule
        Column Name & Meaning & Data Type & Key & Constraint & Domain \\ 
    \midrule
        feeding\_id & 紀錄編號 & int & PK & Not Null & AUTO\_INCREMENT \\ 
        a\_id & 動物編號 & varchar(10) & FK & Not Null & \\ 
        f\_id & 飼料編號 & varchar(10) & FK & Not Null & \\ 
        fed\_by & 操作員工 & varchar(10) & FK & Not Null & \\ 
        amount & 餵食量(kg) & decimal & ~ & Not Null & > 0 \\ 
    \bottomrule
    \end{tabular}}
    \caption{資料表 FEEDING\_RECORDS 的欄位資訊} 
\end{table}

\section{系統實作}

\subsection{資料庫建置方式及資料來源說明}

為了模擬真實且大量的運作情境,本系統的測試資料並非手動輸入,而是透過 Python 的 \verb|Faker| 函式庫與自定義腳本生成。
\begin{itemize}
    \item \textbf{員工資料}:生成 50 位虛擬員工,隨機分配 User 與 Admin 角色及不同性別。
    \item \textbf{動物資料}:建立了包含 20 種物種(如獅子、大象、企鵝等)的資料庫,並為每種動物設定了特定的 \verb|required_skill|(照養門檻)。
    \item \textbf{歷史紀錄}:模擬過去一年的營運數據,生成了超過 2,000 筆餵食紀錄與超過 15,000 筆體重變化紀錄,龐大的體重數據也驗證了我們導入 NoSQL 處理健康警示的必要性。
\end{itemize}

\subsection{重要功能及對應的 SQL 指令}

\subsubsection{給 User 的功能}

\begin{enumerate}
\item \textbf{新增動物體重紀錄}:
當照養員量測完體重後,系統會將數值寫入 PostgreSQL,同時透過 Window Function 計算變化率,若異常則寫入 MongoDB。
在此之前,系統會執行嚴格的 **權限檢查 (Permission Check)**:
\begin{enumerate}
    \item \textbf{排班檢查}:確認該員工目前是否處於值班時間,且該動物是否為其負責對象。
    \item \textbf{技能檢定}:若該動物屬於特殊物種(如企鵝),檢查員工是否持有對應證照。
\end{enumerate}
\begin{lstlisting}[language=SQL, caption={權限檢查 SQL}, captionpos=b]
-- 1. 檢查是否值班中
SELECT shift_id FROM employee_shift
WHERE e_id = %s AND a_id = %s AND NOW() BETWEEN start_time AND end_time;

-- 2. 檢查是否擁有特定技能 (若動物需要)
SELECT skill_id FROM employee_skills 
WHERE e_id = %s AND skill_name = (SELECT required_skill FROM animal WHERE a_id = %s);
\end{lstlisting}

\item \textbf{查詢我的班表}:
使用 Join 查詢員工的排班表以及對應的工作項目。
\begin{lstlisting}[language=SQL, caption={查詢班表 SQL}, captionpos=b]
SELECT s.shift_start, s.shift_end, t.t_name, s.a_id
FROM employee_shift s
JOIN task t ON s.t_id = t.t_id
WHERE s.e_id = %s
ORDER BY s.shift_start DESC LIMIT 10;
\end{lstlisting}
\end{enumerate}

\subsubsection{給 Admin 的功能}

\begin{enumerate}
\item \textbf{庫存管理 (進貨與報表)}:
管理員可查看各類飼料的庫存水位。系統透過聚合函數計算 `feeding_inventory` 表中的所有變動(進貨為正,消耗為負)來得出當前庫存。
\begin{lstlisting}[language=SQL, caption={庫存報表 SQL}, captionpos=b]
SELECT f.feed_name, SUM(i.quantity_delta_kg) as current_stock
FROM feeding_inventory i
JOIN feeds f ON i.f_id = f.f_id
GROUP BY f.feed_name
ORDER BY current_stock ASC;
\end{lstlisting}

\item \textbf{指派工作 (Assign Task)}:
在指派工作前,系統同樣會執行技能檢查,確保不會將需要特殊照護的動物指派給未受訓的員工。
\end{enumerate}

\subsection{交易管理 (Transaction Management)}

在「新增餵食紀錄」功能中,我們嚴格實作了交易管理機制,以確保庫存數據的一致性。
操作流程包含:(1) 鎖定飼料庫存 (2) 寫入餵食紀錄 (3) 扣除庫存。這些步驟必須在同一個 Transaction 中完成,若中途庫存不足或發生錯誤,則全數 Rollback。

我們使用 Python 的 Context Manager 配合 \verb|psycopg2| 的連線池來實作:
\begin{lstlisting}[language=Python, caption={交易管理實作 (Python)}, captionpos=b]
try:
    with self.get_db_connection() as conn:
        cur = conn.cursor()
        # 1. 鎖定並檢查庫存
        cur.execute("SELECT f_id FROM feeds WHERE f_id = %s FOR UPDATE", (f_id,))
        # ... 檢查庫存邏輯 ...
        
        # 2. 寫入紀錄 & 3. 扣除庫存
        cur.execute(insert_feeding_sql, ...)
        cur.execute(update_inventory_sql, ...)
        
        # 成功則自動 Commit
        conn.commit()
except Exception:
    # 失敗則自動 Rollback
    conn.rollback()
\end{lstlisting}

\subsection{併行控制 (Concurrency Control)}

為了防止多位照養員同時領取飼料導致「超賣(庫存變成負數)」,我們使用了 PostgreSQL 的 **悲觀鎖 (Pessimistic Locking)**。

在上述的交易中,我們使用了 \verb|FOR UPDATE| 語句:
\begin{lstlisting}[language=SQL]
SELECT f_id FROM feeds WHERE f_id = 'F001' FOR UPDATE;
\end{lstlisting}
這確保了當一個交易正在計算並扣除庫存時,其他試圖讀取或修改該飼料庫存的交易必須等待,直到前一個交易 Commit 或 Rollback 為止。這有效解決了 Race Condition 問題。

此外,為了防止在高併發寫入時 Primary Key 衝突,我們在寫入前使用了 Table Lock:
\begin{lstlisting}[language=SQL]
LOCK TABLE feeding_records, feeding_inventory IN SHARE ROW EXCLUSIVE MODE;
\end{lstlisting}

\subsection{NoSQL 資料庫整合}

本系統引入 MongoDB 作為輔助資料庫,專門處理具備「高寫入頻率」、「結構多變」或「時序性」特徵的資料。相較於 PostgreSQL 的嚴格綱要,MongoDB 的 Schema-less 特性為日誌與警示系統提供了極大的靈活性。

\subsubsection{稽核日誌 (Audit Logs)}
當管理員或照養員執行敏感操作(如修正過往的餵食或體重紀錄)時,系統會自動記錄一筆稽核日誌。由於不同資料表(如 \verb|feeding_records| 與 \verb|animal_state_record|)的欄位差異甚大,使用關聯式資料庫儲存變更紀錄相當繁瑣。
MongoDB 允許我們將變更前後的數值快照(Snapshot)直接以巢狀 JSON 文件儲存,保留完整的操作軌跡。

\begin{lstlisting}[language=Python, caption={Audit Log JSON 結構範例}, captionpos=b]
{
    "event_type": "DATA_CORRECTION",
    "timestamp": "2025-12-03T14:30:00",
    "operator_id": "E001",
    "target_table": "animal_state_record",
    "record_id": 105,
    "change": {
        "field": "weight",
        "old_value": 150.5,
        "new_value": 155.0
    },
    "original_creator_id": "E005"
}
\end{lstlisting}

\subsubsection{健康警示 (Health Alerts)}
系統內建的分析模組會定期檢查動物的生理數值(例如體重)。一旦偵測到異常波動(如變化率超過 5\%),系統會即時寫入一筆警示紀錄至 \verb|health_alerts| 集合。
我們利用 MongoDB 強大的聚合框架(Aggregation Framework)來實作「高風險動物名單」功能,快速統計特定時間區間內異常次數過多的動物。

\begin{lstlisting}[language=Python, caption={MongoDB 聚合查詢 (找出高風險動物)}, captionpos=b]
pipeline = [
    # 1. 依照動物 ID 分組並計算警示次數
    {"$group": {"_id": "$animal_id", "count": {"$sum": 1}}},
    # 2. 篩選出異常次數 >= 3 次的動物
    {"$match": {"count": {"$gte": 3}}}, 
    # 3. 依照次數由高到低排序
    {"$sort": {"count": -1}}
]
high_risk_animals = db.health_alerts.aggregate(pipeline)
\end{lstlisting}



\subsection{跨資料庫整合應用案例:冒失鬼名單 (Careless Employees)}

本系統的一大特色是結合 SQL 與 NoSQL 的優勢,實作了「冒失鬼名單」分析功能。此功能旨在找出輸入錯誤率過高(經常被修正紀錄)的員工,以利管理員進行教育訓練。

此功能的實作流程展示了跨資料庫的資料整合:
\begin{enumerate}
    \item \textbf{NoSQL 聚合分析}:首先在 MongoDB 的 \verb|audit_logs| 中,篩選出 \verb|DATA_CORRECTION| 事件,並依照原始建立者 (\verb|original_creator_id|) 分組統計被修正的次數。我們設定閾值(例如 5 次),篩選出錯誤頻率過高的員工 ID。
    \item \textbf{SQL 資料豐富化 (Enrichment)}:取得上述的員工 ID 列表後,系統會查詢 PostgreSQL 的 \verb|employee| 資料表,獲取該員工的真實姓名與職稱,將冰冷的 ID 轉換為具可讀性的報表。
\end{enumerate}

\begin{lstlisting}[language=Python, caption={冒失鬼名單實作邏輯}, captionpos=b]
# 1. MongoDB: 找出被修正次數 >= 5 的員工 ID
pipeline = [
    {"$match": {"event_type": "DATA_CORRECTION"}},
    {"$group": {"_id": "$original_creator_id", "count": {"$sum": 1}}},
    {"$match": {"count": {"$gte": 5}}},
    {"$sort": {"count": -1}}
]
careless_list = db.audit_logs.aggregate(pipeline)

# 2. PostgreSQL: 查詢員工姓名並整合結果
results = []
with self.get_db_connection() as conn:
    cur = conn.cursor()
    for item in careless_list:
        e_id = item['_id']
        # 跨資料庫查詢員工姓名
        cur.execute("SELECT e_name FROM employee WHERE e_id = %s", (e_id,))
        name = cur.fetchone()[0]
        results.append({"name": name, "error_count": item['count']})
\end{lstlisting}

\section{分工資訊}
% 請自行填寫

\section{專案心得}
% 請自行填寫

\end{document}
